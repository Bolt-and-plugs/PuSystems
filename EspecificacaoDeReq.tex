\documentclass[a5paper, 12pt]{article}

\usepackage[portuges]{babel}
\usepackage[utf8]{inputenc}
\usepackage[T1]{fontenc}
\usepackage{times} % Fonte Times New Roman
\usepackage{setspace}
\usepackage{mathptmx} 
\onehalfspacing
\usepackage{amsmath}
\usepackage{indentfirst}
\usepackage{graphicx}
\usepackage{multicol,lipsum}
\usepackage{enumitem}
\usepackage{url}
\usepackage{booktabs}
\usepackage{array}
\usepackage{tocloft}
\usepackage{fancyhdr}
\usepackage{tabularx}   
\usepackage{longtable}
\usepackage{float}
\usepackage{booktabs}   
\usepackage{makecell}
\usepackage[a4paper, 
            left=3cm,    
            right=2cm,   
            top=3cm,    
            bottom=2cm   
            ]{geometry}


\begin{document}
%\maketitle

\begin{titlepage}
	\begin{center}
	
	\begin{figure}[!ht]
	\centering
	\includegraphics[width=10cm]{dist/LogoTransparentePreto.png} \\ 
    \end{figure}

        
		\vspace{115pt}
        \textbf{\Huge{Especificação de Requisitos}}\\
		%\title{{\large{Título}}}
        
		\vspace{115pt}
        Carlos Eduardo Nogueira Silva \\
        Felipe Gomes da Silva \\
        Felipe Matheus Possari \\
        Matheus Thomé da Silva\\ 
        Santiago Pinheiro Martins \\
	\end{center}
	
	
	\vspace{1cm}
	\begin{center}
		\vspace{\fill}
		 Abril \\
		 2025
			\end{center}
\end{titlepage}
%%%%%%%%%%%%%%%%%%%%%%%%%%%%%%%%%%%%%%%%%%%%%%%%%%%%%%%%%%%




\newpage
\thispagestyle{empty}
\tableofcontents

\newpage
\pagestyle{fancy}

\fancyhead[L]{\thepage}
\fancyhead[C]{\nouppercase{\leftmark}}
\rhead{\includegraphics[width=2cm]{dist/LogoTransparentePreto.png}}
\fancyfoot[R]{}
\fancyfoot[C]{© 2025 - PuSystems}
\fancyfoot[L]{}
\setlength\headheight{26pt}

% % % % % % % % % % % % % % % % % % % % % % % % % % %
\section{Introdução}

São José do Rio Preto está lidando com uma das mais graves epidemias de dengue já registradas no estado, com mais de 35.000 casos confirmados da enfermidade apenas em 2025, posicionando-se como a cidade com maior probabilidade de casos prováveis em todo o Brasil. Claro que isso acontece, mesmo com os esforços das autoridades públicas para conscientizar a população, implementar uma urbanização adequada e eliminar focos do mosquito da dengue.

Isso destaca a importância da população na luta contra o mosquito, sendo o único grupo capaz de identificar facilmente os focos de dengue em sua residência e arredores. Então, nosso objetivo é capacitar os habitantes de São José do Rio Preto para que eles mesmos possam aprimorar a condição da cidade e se empenhem para isso.

Para estimular a participação, o sistema se baseará no modelo de \textit{gamificação} \textit{Octalysis}, criado pelo escritor e empresário Yu-Kai Chou.  O modelo \textit{Octalysis} é empregado na criação de sistemas de \textit{gamificação} destinados a potencializar a motivação humana, melhor tratado na seção \ref{sec:game}.

Assim, nosso objetivo é desenvolver um sistema que incentive todos esses elementos motivacionais na população de São José do Rio Preto, permitindo que eles se conscientizem sobre sua própria saúde, entendam o ciclo de vida do Aedes aegypti e do vírus da dengue, além de agirem, eliminando focos de proliferação e acionando as autoridades em situações de emergência. Cada um dos 8 focos estará associado a requisitos funcionais tratados na seção \ref{sec:req}.

\section{O sistema e seu funcionamento}

O sistema a ser desenvolvido será  um aplicativo que funcionará como uma plataforma unificada, combinando informações geoespaciais, dados climáticos e registros de casos de dengue para reconhecer e acompanhar zonas de perigo, tanto para as autoridades quanto para a população. O sistema determina uma pontuação para cada bairro, possibilitando a monitorização instantânea da situação local. Os usuários têm a capacidade de reportar pontos de proliferação de mosquitos, que serão examinados e erradicados pelas autoridades. Cada queixa confirmada e solucionada eleva a pontuação do usuário, que pode ser transformada em prêmios. Adicionalmente, ao aprimorar a condição de um bairro, a pontuação dessa região também é elevada.

O aplicativo terá um componente de registro e administração de usuários, possibilitando a criação de um perfil personalizado com dados como nome social, documento oficial, foto de perfil, gênero, idade, local de residência e ênfase no combate à dengue. Os usuários poderão modificar suas informações e definir preferências, tais como notificações e categorias de missões. Será preciso fazer login para acessar recursos relevantes, assegurando a proteção dos dados através de boas práticas de criptografia.

No segmento educacional, serão disponibilizados conteúdos interativos, tais como aulas em vídeo, transmissões ao vivo sobre a luta contra o mosquito, além de recursos de suporte, tais como infográficos, questionários e textos explicativos sobre o ciclo de vida do Aedes aegypti, os sintomas da dengue e métodos de prevenção.

\subsection{Gamificação}
\label{sec:game}
A \textit{gamificação} será um elemento fundamental do aplicativo, com um sistema de pontuação individual e classificação global, além de classificação específica por bairro, promovendo a competição sadia. O sistema também contemplará realizações pessoais, tais como "Denúncias Realizadas" e "Focos Eliminados", bem como missões individuais e coletivas, premiando ações executadas e penalizando a inatividade. Missões de duração limitada e prêmios inesperados acrescentarão um elemento adicional de envolvimento e urgência.

Para alcancar este objetivo, utilizaremos o sistema de gamificacao Octalysis que alcança nosso objetivo de engajamento ao reconhecer oito "impulsos principais" que precisam ser estimulados, sao elas:

\begin{enumerate}
    \item \textbf{Epic Meaning \& Calling} se refere à motivação em participar de algo que consideramos maior que nós mesmos, tomando partido em uma causa nobre ou propósito grandioso.
    \item \textbf{Development \& Accomplishment}Refere-se à motivação que obtemos ao avançar em direção a algum objetivo e superarmos obstáculos e alcançarmos marcos.
    \item \textbf{Empowerment of Creativity \& Feedback} é a motivação que surge da chance de criar, experimentar e observar as consequências de suas ações.
    \item \textbf{Ownership \& Possession} é o impulso que temos ao percebermos que possuímos responsabilidade sobre algo, seja um projeto, ideia ou objetivo.
    \item \textbf{Social Influence \& Relatedness} é a motivação social e ocorre através do contato com outras pessoas.
    \item \textbf{Scarcity \& Impatience} está intimamente ligado à percepção de valor que aumenta quando a disponibilidade de um item ou recurso é restrita.
    \item \textbf{Unpredictability \& Curiosity} explora a imprevisibilidade presente em uma experiência, como surpresas e enigmas a serem desvendados, como uma forte fonte de motivação.
    \item \textbf{Loss \& Avoidance} é a motivação que origina da vontade de evitar danos ou prejuízos.
\end{enumerate}

O sistema de alertas e acompanhamento de zonas de risco possibilitará que os usuários reportem focos de mosquitos, sendo as denúncias confirmadas pela equipe encarregada. As informações serão empregadas na criação de um mapa de zonas de risco, mostrando as áreas com maior incidência de focos e casos de dengue, com atualizações em tempo real sobre o avanço das medidas de combate.

Ademais, o sistema de rastreamento de sintomas possibilitará aos usuários o preenchimento de um formulário diário sobre seus sintomas, com notificações em caso de suspeita de dengue. O registro de saúde do usuário auxiliará no monitoramento da progressão do seu estado de saúde e fornecerá alertas de monitoramento.

\subsection{Recompensas para o Usuário}

Para aumentar o envolvimento da população, o sistema oferecerá um conjunto de recompensas que estão em conformidade com os princípios do modelo Octalysis. As recompensas serão distribuídas com base no nível de envolvimento, pontuação acumulada e conclusão de missões, incentivando a motivação constante e a sensação de avanço.

As recompensas serão divididas em dois grupos principais:

\begin{itemize}
    \item \textbf{Recompensas simbólicas e sociais:}
    \begin{itemize}
        \item Certificados digitais de participação e impacto social.
        \item Conquistas visuais no perfil do usuário (badges, medalhas).
        \item Destaque em rankings públicos do aplicativo.
        \item Acesso a conteúdos e funcionalidades exclusivas dentro da plataforma.
    \end{itemize}

    \item \textbf{Recompensas materiais e promocionais (parcerias privadas):}
    \begin{itemize}
        \item Brindes como camisetas, garrafas reutilizáveis ou kits educativos.
        \item Cupons de desconto em empresas parceiras locais (mercados, farmácias, transporte), ou com sistemas que a administração atual julgue possível e importantes.
        \item Participação em sorteios mensais com prêmios patrocinados.
    \end{itemize}
\end{itemize}

Essas recompensas visam não apenas incentivar ações individuais, mas também reforçar o senso de comunidade, pertencimento e colaboração no combate à dengue.

\section{Os requisitos funcionais}
\label{sec:req}

Para que o sistema atinja suas metas e proporcione uma experiência eficiente para os usuários, deve-se implementar uma série de requisitos funcionais que vão desde o registro de usuários até a interação com autoridades e mecanismos de gamificação.

\subsection{Registro e Gestão de Usuários}
\begin{enumerate}
    \item Registro de um perfil personalizado, com os seguintes campos: nome social, documento oficial, foto de perfil, gênero, idade, endereço (bairro) e foco no combate à dengue.
    \item Exigência de login com senha e documento oficial e senha para acesso aos recursos do aplicativo.
    \item Possibilidade de atualização dos dados cadastrais.
\end{enumerate}

\subsection{Módulo Educacional}
\begin{enumerate}
    \item Conteúdo organizado em diferentes cursos/módulos para melhor absorção e aprendizado.
    \item \textit{Player} de vídeos para vídeo-aulas, com suporte a legendas.
    \item Transmissões ao vivo para anúncios oficiais em palestras de temas relevantes
    \item Tutoriais multimídias (textos, vídeos e \textit{quizzes}) que ensinam a identificar e eliminar focos de mosquito da dengue.
\end{enumerate}

\subsection{Sistema de Gamificação (Baseado no Octalysis)}
Os princípios do modelo Octalysis serão utilizados para orientar a gamificação no sistema, que identifica oito impulsos motivacionais principais que envolvem os usuários em variados contextos. Essas motivações serão integradas às características do aplicativo para incentivar a participação engajada, o sentido de propósito, a competição saudável e recompensas relevantes.
\begin{enumerate}
    \item Score pessoal baseado em ações como denúncias e consumo de conteúdo educativo (Development \& Accomplishment*).
    \item Rankings global e por bairro (Social Influence \& Relatedness).
    \item Destaque para os usuários mais ativos na página inicial.
    \item Conquistas pessoais desbloqueadas por frequência e impacto das ações realizadas.
    \item Histórico de atividades e conquistas acessível pelo perfil.
    \item Missões individuais aleatórias e personalizadas com base nos hábitos do usuário (Empowerment of Creativity \& Feedback).
    \item Missões cooperativas para estimular ações comunitárias (Epic Meaning \& Calling).
    \item Recompensas por ações específicas e conclusão de missões.
    \item Penalidades leves por inatividade prolongada (Loss \& Avoidance).
    \item Recompensas surpresa por interação com o app (Unpredictability \& Curiosity).
    \item Missões de tempo limitado que geram senso de urgência (Scarcity \& Impatience).
\end{enumerate}



\subsection{Sistema de Denúncias e Monitoramento de Áreas de Risco}
\begin{enumerate}
    \item Formulário de denúncia de focos de mosquito, incluindo texto, foto e localização.
    \item Envio automático de denúncias para a equipe responsável para que validem e atualizem o status do foco. 
    \item Exibição de áreas de risco no mapa, utilizando dados climáticos, registros de casos e localização de denúncias para indicar regiões com maior incidência.
    \item Envio de notificações ao usuário sobre aumento de casos em sua região ou surgimento de focos próximos ao seu endereço.
    \item Envio de notificações sobre a pontuação do bairro, informando se está melhorando ou piorando com base no engajamento comunitário.
\end{enumerate}

\subsection{Sistema de Monitoramento de Sintomas}
\begin{enumerate}
    \item Um questionário simples com perguntas sobre febre, dor de cabeça, dores musculares e outros sintomas característicos da dengue que pode ser preenchido periodicamente
    \item Alertas de suspeita, caso os sintomas sejam compatíveis com a doença, indicando a necessidade de buscar atendimento médico e informando as autoridades de saúde.
\end{enumerate}
\subsection{Gerenciamento e Suporte a Autoridades}
\begin{enumerate}
    \item Painel de estatísticas permitindo que autoridades monitorem as informações geoespaciais da cidade e bairros, as denúncias de focos e o número de casos  reportados
    \item Painel para a avaliação e atualização dos status de denúncias realizadas.
    \item Integração com sistema oficiais para que dados sobre surtos, casos e relatórios de saúde pública possam ser importados.
\end{enumerate}




\section{Conclusão}
O sistema proposto tem como objetivo combinar tecnologia, educação e envolvimento social no combate à dengue em São José do Rio Preto. O uso do modelo de \textit{gamificação Octalysis} assegura uma motivação completa para os usuários, abrangendo elementos emocionais, sociais, criativos e racionais. Esperamos não só amenizar a crise atual ao motivar a população através de recompensas, desafios e aprendizado constante, mas também promover uma cultura de prevenção e envolvimento cidadã sustentável a longo prazo.

\end{document}
