\documentclass[a5paper, 12pt]{article}

\usepackage[portuges]{babel}
\usepackage[utf8]{inputenc}
\usepackage[T1]{fontenc}
\usepackage{times}
\usepackage{setspace}
\usepackage{mathptmx} 
\onehalfspacing
\usepackage{amsmath}
\usepackage{indentfirst}
\usepackage{graphicx}
\usepackage{multicol}
\usepackage{enumitem}
\usepackage{url}
\usepackage{booktabs}
\usepackage{array}
\usepackage{tocloft}
\usepackage{fancyhdr}
\usepackage{tabularx}   
\usepackage{longtable}
\usepackage{float}
\usepackage{booktabs}   
\usepackage{makecell}
\usepackage{xcolor}
\usepackage{hyperref}
\usepackage{amssymb}
\usepackage[a4paper, 
            left=3cm,    
            right=2cm,   
            top=3cm,    
            bottom=2cm   
            ]{geometry}

\begin{document}

\begin{titlepage}
	\begin{center}
	
	\begin{figure}[!ht]
	\centering
	% \includegraphics[width=10cm]{dist/LogoTransparentePreto.png} \\ 
    \end{figure}

	\vspace{115pt}
    \textbf{\Huge{Projeto de Interface}}\\
        
	\vspace{115pt}
    Carlos Eduardo Nogueira Silva \\
    Felipe Gomes da Silva \\
    Felipe Matheus Possari \\
    Matheus Thomé da Silva\\ 
    Santiago Pinheiro Martins \\
	\end{center}
	
	\vspace{1cm}
	\begin{center}
		\vspace{\fill}
		Abril \\
		2025
	\end{center}
\end{titlepage}

\newpage
\thispagestyle{empty}
\tableofcontents

\newpage
\pagestyle{fancy}
\fancyhead[L]{\thepage}
\fancyhead[C]{\nouppercase{\leftmark}}
% \rhead{\includegraphics[width=2cm]{dist/LogoTransparentePreto.png}}
\fancyfoot[R]{}
\fancyfoot[C]{© 2025 - PuSystems}
\fancyfoot[L]{}
\setlength\headheight{26pt}

\section{Introdução}
Este documento apresenta o projeto de interface para o aplicativo ZAPP - Zona de Ação Pública, desenvolvido pela PuSystems. O objetivo é fornecer uma análise detalhada das telas do sistema, aplicando os princípios de usabilidade e design de interface para criar uma experiência intuitiva e eficiente para os usuários no combate à dengue.

\section{Desenvolvimento}

\subsection{Análise}
A análise das telas foi realizada com base nos princípios de usabilidade de Pressman (Capítulo 11), considerando:
\begin{itemize}
\item Perfil dos usuários (novatos, intermediários e avançados)
\item Tarefas que os usuários precisam realizar
\item Ambiente físico de uso do aplicativo
\item Fatores humanos como memória limitada e propensão a erros
\end{itemize}

\subsection{Projeto}
\subsubsection{Tela de Login}
\textbf{Princípios Aplicados:}
\begin{itemize}[leftmargin=*]
    \item \textbf{Deixar o usuário no comando:}
    \begin{itemize}
        \item Flexibilidade de escolha entre login tradicional e autenticação via gov.br
        \item Opção de recuperação de acesso
    \end{itemize}

    \item \textbf{Detalhes:}
    \begin{itemize}
        \item Ícone de olho para mostrar/ocultar senha
        \item Mapa na parte inferior da tela para mostrar identidade visual
    \end{itemize}
\end{itemize}

\subsubsection{Tela de Cadastro}
\textbf{Princípios Aplicados:}
\begin{itemize}[leftmargin=*]  
    \item \textbf{Detalhes:}
    \begin{itemize}
        \item Opção de autenticação alternativa (gov.br)
        \item Separação visual clara entre opções
    \end{itemize}

    \item \textbf{Experiência do usuário:}
    \begin{itemize}
        \item Link para visualização completa dos termos de uso e políticas de privacidade
        \item Validação em tempo real dos campos
        \item Dicas de requisitos para senha
    \end{itemize}
\end{itemize}

\subsubsection{Tela Home}
\textbf{Princípios Aplicados:}
\begin{itemize}[leftmargin=*]
    \item \textbf{Navegação simplificada:}
    \begin{itemize}
        \item Botões principais destacados
        \item Ícones autoexplicativos
    \end{itemize}

    \item \textbf{Detalhes:}
    \begin{itemize}
        \item Área de notificações
        \item Seção de missões
    \end{itemize}
\end{itemize}

\subsubsection{Tela Meu Bairro}
\textbf{Princípios Aplicados:}
\begin{itemize}[leftmargin=*]
    \item \textbf{Orientação espacial:}
    \begin{itemize}
        \item Campo de pesquisa por região
        \item Listagem de áreas próximas
    \end{itemize}
    
    \item \textbf{Informação contextual:}
    \begin{itemize}
        \item Seção dedicada para alertas na parte inferior
    \end{itemize}

    \item \textbf{Detalhes:}
    \begin{itemize}
        \item Mapa interativo integrado
        \item Sistema de cores para prioridade de alertas
    \end{itemize}
\end{itemize}

\subsubsection{Tela Denunciar Foco}
\textbf{Princípios Aplicados:}
\begin{itemize}[leftmargin=*]
    \item \textbf{Localização automática:}
    \begin{itemize}
        \item Detecção de GPS
        \item Opção de ajuste manual
    \end{itemize}

    \item \textbf{Detalhes:}
    \begin{itemize}
        \item Upload de fotos integrado
        \item Confirmação de envio com feedback
    \end{itemize}
\end{itemize}

\subsubsection{Tela Monitoramento de Sintomas}
\textbf{Princípios Aplicados:}
\begin{itemize}[leftmargin=*]
    \item \textbf{Checklist intuitivo:}
    \begin{itemize}
        \item Sintomas comuns listados
        \item Checklist para selecionar o que está sentindo
    \end{itemize}
\end{itemize}

\subsubsection{Tela Aprender}
\textbf{Princípios Aplicados:}
\begin{itemize}[leftmargin=*]
    \item \textbf{Conteúdo educacional:}
    \begin{itemize}
        \item Materiais variados, vídeos, artigos e apostílas
        \item Destaque para informações importantes
    \end{itemize}

    \item \textbf{Detalhes:}
    \begin{itemize}
        \item Busca por tópicos
        \item Progresso de aprendizagem
        \item Botão de salvar acima da barra de pesquisa
        \item Player de reprodução e opção de compartilhar
    \end{itemize}
\end{itemize}

\subsubsection{Tela Missões}
\textbf{Princípios Aplicados:}
\begin{itemize}[leftmargin=*]
    \item \textbf{Gamificação:}
    \begin{itemize}
        \item Sistema de pontos e recompensas
        \item Barras de progresso para as missões
    \end{itemize}
    
    \item \textbf{Detalhes:}
    \begin{itemize}
        \item Pontuação visual
        \item Sistema de conquistas
    \end{itemize}
\end{itemize}

\subsubsection{Tela de Loading}
\textbf{Princípios Aplicados:}
\begin{itemize}[leftmargin=*]
    \item \textbf{Feedback visual:}
    \begin{itemize}
        \item Indicador de carregamento com ícone piscando
        \item Mapa no fundo, em verde, mostrando a identidade visual do aplicativo
    \end{itemize}
\end{itemize}

\subsection{Construção}
A construção da interface seguiu um processo iterativo com:
\begin{itemize}
\item Prototipação no papel para validação inicial
\item Desenvolvimento de protótipos funcionais
\item Testes com usuários reais
\item Refinamento contínuo baseado em feedback
\end{itemize}

\subsection{Validação}
A validação foi realizada através de:
\begin{itemize}
\item Testes de usabilidade com diferentes perfis de usuários
\item Coleta de dados quantitativos (tempo de tarefa, erros)
\item Avaliação qualitativa via questionários
\item Ajustes finais baseados nos resultados
\end{itemize}

\section{Conclusão}
O projeto de interface desenvolvido para o aplicativo ZAPP atende aos princípios de usabilidade estabelecidos, proporcionando uma experiência intuitiva e eficiente para os usuários no combate à dengue. As telas foram projetadas considerando a diversidade de usuários e as diferentes tarefas que precisam realizar, com foco especial na redução da carga cognitiva e na prevenção de erros.

\begin{thebibliography}{09}
\bibitem{ibge-2022} INSTITUTO BRASILEIRO DE GEOGRAFIA E ESTATÍSTICA (IBGE). 
\textbf{Censo Demográfico 2022: São José do Rio Preto}. 
2022. 
Disponível em: \url{https://www.ibge.gov.br/cidades-e-estados/sp/sao-jose-do-rio-preto.html}. 
Acesso em: 20 abr. 2025.

\bibitem{pressman} PRESSMAN, R. S. 
\textbf{Engenharia de Software}. 
8. ed. McGraw Hill, 2019.

\bibitem{nielsen} NIELSEN, J. 
\textbf{10 Usability Heuristics for User Interface Design}. 
Nielsen Norman Group, 1994.
\end{thebibliography}

\end{document}
