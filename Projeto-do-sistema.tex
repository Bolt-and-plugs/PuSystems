\documentclass[a4paper, 12pt]{article}
\usepackage[portuguese,provide=*]{babel}
\usepackage[utf8]{inputenc}
\usepackage[T1]{fontenc}
\usepackage{times} % Fonte Times New Roman
\usepackage{setspace}
\usepackage{mathptmx} 
\onehalfspacing
\usepackage{amsmath}
\usepackage{indentfirst}
\usepackage{graphicx}
\usepackage{multicol,lipsum}
\usepackage{enumitem}
\usepackage{url}
\usepackage{booktabs}
\usepackage{array}
\usepackage{tocloft}
\usepackage{fancyhdr}
\usepackage{tabularx}   
\usepackage{longtable}
\usepackage{float}
\usepackage{booktabs}   
\usepackage{makecell}
\usepackage[a4paper, 
            left=3cm,    
            right=2cm,   
            top=3cm,    
            bottom=2cm   
            ]{geometry}


\begin{document}
%\maketitle

\begin{titlepage}
	\begin{center}
	
	\begin{figure}[!ht]
	\centering
	\includegraphics[width=10cm]{dist/LogoTransparentePreto.png} \\ 
    \end{figure}

        
		\vspace{115pt}
        \textbf{\Huge{Projeto do Sistema}}\\
		%\title{{\large{Título}}}
        
		\vspace{115pt}
        Carlos Eduardo Nogueira Silva \\
        Felipe Gomes da Silva \\
        Felipe Matheus Possari \\
        Matheus Thomé da Silva\\ 
        Santiago Pinheiro Martins \\
	\end{center}
	
	
	\vspace{1cm}
	\begin{center}
		\vspace{\fill}
		 Abril \\
		 2025
			\end{center}
\end{titlepage}
%%%%%%%%%%%%%%%%%%%%%%%%%%%%%%%%%%%%%%%%%%%%%%%%%%%%%%%%%%%


\newpage
\thispagestyle{empty}
\tableofcontents

\newpage
\pagestyle{fancy}

\fancyhead[L]{\thepage}
\fancyhead[C]{\nouppercase{\leftmark}}
\rhead{\includegraphics[width=2cm]{dist/LogoTransparentePreto.png}}
\fancyfoot[R]{}
\fancyfoot[C]{© 2025 - PuSystems}
\fancyfoot[L]{}
\setlength\headheight{26pt}

% % % % % % % % % % % % % % % % % % % % % % % % % % %


\newpage
\section{Introdução}

O \textit{Zapp} surge como uma ponte entre cidadãos e autoridades no combate à dengue, sendo mais do que um aplicativo, a solução propõe uma experiência interativa, transformando a prevenção em uma jornada coletiva. Com módulos que vão desde gamificação até monitoramento de sintomas (melhor exemplificado no documento Plano do Projeto), a plataforma busca engajar usuários por meio de recompensas, educação acessível e respostas ágeis a riscos epidemiológicos.

A arquitetura do sistema, pensada para ser dinâmica e escalável para os moradores de São José do Rio Preto, integra ferramentas modernas como \textit{React Native, Node.js} e \textit{PostgreSQL}, garantindo não apenas segurança e eficiência, mas também adaptabilidade para crescimento. Este documento explora como cada decisão, arquitetura e componente — da autenticação segura às integrações com serviços externos — se conectam para criar um ecossistema digital capaz de salvar vidas.

\newpage
\section{Desenvolvimento}

\subsection{Modelagem do Sistema}
A modelagem do Zapp evidencia a complexidade e a multidisciplinaridade requeridas para unir cidadãos, autoridades e tecnologia numa plataforma focada na luta contra a dengue. A arquitetura do sistema foi estruturada com base nos diagramas apresentados para assegurar a participação coletiva, rapidez na resposta epidemiológica e escalabilidade, em conformidade com os pilares de interatividade, segurança e adaptabilidade descritos no projeto. A seguir, diversos diagramas serão utilizados para melhor entendimento acerca do projeto.


\subsubsection{Diagrama de Classes}
O Diagrama de Classes (Figura \ref{fig:classes}) estabelece a espinha dorsal do sistema, definindo entidades como Usuário, Denúncia, Sintoma, Recompensa e Relatório, além de suas relações. Essa estrutura assegura a gestão de dados essenciais, como perfis de usuários, histórico de interações, denúncias validadas e métricas de gamificação, consolidando a base para operações seguras e eficientes no PostgreSQL.


\begin{figure}[H]
    \centering
    \includegraphics[width=1\textwidth]{dist/Diagrama de clases.png}
    \caption{Diagrama de classes}
    \label{fig:classes}
\end{figure}


\subsubsection{Diagramas de Interação}
Os Diagramas de Interação a seguir detalham processos críticos do sistema

\begin{enumerate}
  \item Geração de Relatórios para Autoridades (Figura \ref{fig:autoridades}): Automatiza a compilação de dados epidemiológicos (ex.: focos de dengue, sintomas reportados) em relatórios acionáveis, facilitando decisões em tempo real pelas autoridades.

  \item Validação de Denúncias (Figura \ref{fig:denuncias}): Define o fluxo colaborativo entre cidadãos (envio de denúncias) e agentes de saúde (análise in loco), garantindo confiabilidade nas informações.

  \item Questionário de Sintomas (Figura \ref{fig:sintomas}): Orienta usuários na autoavaliação de saúde, com respostas direcionando alertas para o sistema de vigilância.

  \item Gamificação (Figura \ref{fig:gamefy}): Conecta conquistas (ex.: eliminação de focos) a recompensas (pontos, badges), incentivando participação contínua.

  \item Integração com APIs Externas (Figura \ref{fig:apis}): Permite interoperabilidade com serviços como mapas georreferenciados e sistemas de saúde públicos, ampliando a capacidade de análise contextual.
\end{enumerate}

\begin{figure}[H]
    \centering
    \includegraphics[width=1\textwidth]{dist/Geração de relatório para autoridades.png}
    \caption{Diagrama de interação para a geração de relatório para autoridades}
    \label{fig:autoridades}
\end{figure}

\begin{figure}[H]
    \centering
    \includegraphics[width=1\textwidth]{dist/Validação de denúncia.png}
    \caption{Diagrama de interação para a validação de denúncias}
    \label{fig:denuncias}
\end{figure}

\begin{figure}[H]
    \centering
    \includegraphics[width=1\textwidth]{dist/Preenchimento de questionário de sintomas.png}
    \caption{Diagrama de interação para o preenchimento do questionário de sintomas}
    \label{fig:sintomas}
\end{figure}

\begin{figure}[H]
    \centering
    \includegraphics[width=1\textwidth]{dist/Sistema de gamificação.png}
    \caption{Diagrama de interação para o sistema de gamificação}
    \label{fig:gamefy}
\end{figure}

\begin{figure}[H]
    \centering
    \includegraphics[width=1\textwidth]{dist/Integração com APIs externas.png}
    \caption{Diagrama de interação para as APIs externas}
    \label{fig:apis}
\end{figure}

\subsubsection{Diagrama de Casos de Uso}
O Diagrama de Casos de Uso (Figura \ref{fig:use-case}) sintetiza as interações entre atores (cidadãos, autoridades, sistemas externos) e funcionalidades centrais, reforçando a natureza colaborativa da plataforma.

\begin{figure}[H]
    \centering
    \includegraphics[width=1\textwidth]{dist/Diagrama de casos de uso.png}
    \caption{Diagrama de casos de uso}
    \label{fig:use-case}
\end{figure}

A união desses modelos demonstra como o Zapp harmoniza a experiência do usuário (por meio da gamificação e do design intuitivo no React Native) com a solidez técnica (com o Node.js cuidando da lógica de negócio e das integrações). A proteção de dados sensíveis é garantida pela autenticação segura no diagrama de classes, enquanto a modularidade da arquitetura possibilita a adaptação a novas necessidades, como a expansão para outras enfermidades ou regiões. Por enquanto, podemos nos concentrar na arquitetura do sistema.

\newpage
\subsection{Projeto de Arquitetura do Sistema}

\subsubsection{Proposta de Arquitetura}

A arquitetura sugerida utiliza uma estratégia cliente-servidor em camadas, incorporando componentes de microsserviços para assegurar escalabilidade, adaptabilidade e alta disponibilidade nos módulos vitais do sistema. Esta estrutura tem como objetivo proporcionar aos usuários uma experiência sólida, com uma integração suave entre as diversas camadas e componentes.

\subsubsection{Cliente (Front)}
 
 O desenvolvimento do lado do cliente é realizado com React Native, uma tecnologia de desenvolvimento multiplataforma que possibilita a produção de aplicativos nativos para Android e iOS, assegurando eficácia e uma experiência de usuário consistente.  O desenvolvedor de interface será encarregado de interagir diretamente com os usuários, oferecendo uma interface intuitiva e de fácil utilização.  A \textbf{autenticação} é um dos módulos principais do frontend, que emprega o \textbf{OAuth2} para um acesso seguro e incorpora um processo de verificação de documentos oficiais, assegurando que somente usuários autenticados tenham acesso às funcionalidades da plataforma.
 
O módulo \textbf{educacional} inclui um reprodutor de vídeos com funcionalidades de legendas e Libras, bem como testes interativos para estimular o aprendizado.  O recurso de \textbf{denúncias} possibilita que os usuários preencham um formulário com a opção de enviar fotos e registrar sua localização geográfica, além de fornecer um retorno para uso offline, assegurando a acessibilidade em regiões com conexão de internet restrita.  A \textbf{gamificação} é realizada através de um sistema de classificação, tarefas e prêmios visuais, fundamentados no modelo Octalysis, promovendo a participação ativa e o envolvimento constante dos usuários.  O \textbf{monitoramento de sintomas} disponibiliza um questionário interativo com a habilidade de emitir alertas em tempo real, possibilitando que o usuário seja notificado imediatamente sobre qualquer sintoma relevante.

\subsubsection{Servidor (Backend)}
 O servidor é construído combinando Node.js e Express.js, resultando em uma API REST sólida e expansível para gerenciar as solicitações do frontend de maneira eficaz.  A estrutura do servidor utiliza containers com Docker, garantindo isolamento, portabilidade e escalabilidade simples.  O \textbf{API Gateway}, que emprega \textbf{NGINX}, tem a função de equilibrar a carga e encaminhar as solicitações, assegurando a alta disponibilidade e distribuindo eficientemente as solicitações entre os microsserviços.  O backend é formado por microsserviços especializados, cada um com uma função crucial a desempenhar.
 
 O microsserviço de \textbf{usuários} gerencia a autenticação, o gerenciamento de perfis e as interações relacionadas aos usuários. O microsserviço de \textbf{gamificação} é responsável por calcular os algoritmos de pontuação, além de aplicar um sistema de \textbf{anti-viés socioeconômico}, garantindo que os processos sejam justos e imparciais para todos os participantes. O microsserviço de \textbf{denúncias} valida as informações recebidas, utilizando \textbf{inteligência artificial} para análise de fotos e integrando-se com sistemas de mapas para georreferenciamento.

subsubsection{Camada de Dados}

A camada de dados é composta por um banco de dados \textbf{PostgreSQL}, que é responsável por armazenar dados estruturados, como as informações dos usuários e as denúncias realizadas. O \textbf{Redis} é utilizado como sistema de cache, otimizando o desempenho ao armazenar rankings e métricas de engajamento, que são acessados com frequência. O armazenamento de arquivos grandes, como fotos e vídeos educacionais, é feito através do serviço \textbf{Amazon S3}, que oferece escalabilidade e segurança para armazenar grandes volumes de dados de forma eficiente e econômica.

\subsubsection{Integrações Externas}

A arquitetura proposta também inclui diversas integrações com serviços externos para enriquecer a funcionalidade da aplicação. A \textbf{OpenWeatherMap} é utilizada para fornecer dados climáticos em tempo real, permitindo que a plataforma faça previsões sobre áreas de risco e condições climáticas que podem afetar a saúde dos usuários. A integração com o \textbf{Google Maps} permite o georreferenciamento de denúncias, facilitando a localização exata dos eventos reportados. Além disso, a integração com a \textbf{Secretaria Municipal de Saúde} permite acessar dados epidemiológicos atualizados, proporcionando informações críticas em tempo real para o monitoramento de possíveis surtos e a adoção de medidas preventivas.

Esta arquitetura foi planejada para garantir alta disponibilidade, escalabilidade e desempenho, criando uma plataforma robusta e eficiente que atende tanto às necessidades dos usuários quanto às exigências de processamento de dados e integração com sistemas externos.

\subsubsection{Diagrama da Arquitetura}

A seguir, apresentamos o diagrama que ilustra a arquitetura proposta, mostrando as camadas e seus componentes principais.

\begin{figure}[htbp]
    \centering
    \includegraphics[width=0.8\textwidth]{dist/Arquitetura final.png}
    \caption{Diagrama da Arquitetura usada no sistema Zapp}
    \label{fig:arquitetura}
\end{figure}


\subsection{Organização dos Componentes}

O Zapp será estruturado em seis componentes principais, cada um encarregado de uma área funcional específica, porém de forma interligada.  Esta estrutura modular visa assegurar a escalabilidade, simplificar a manutenção e possibilitar a evolução constante do sistema.

\subsubsection{Módulo de Autenticação e Gestão de Perfis}

Este módulo será encarregado de todas as etapas de acesso e identificação dos usuários.  Os usuários terão a possibilidade de criar perfis personalizados, fornecendo informações como nome, idade, sexo, localização (bairro), documento de identidade e uma imagem.  O sistema solicitará a autenticação através de uma senha e um documento válido, utilizando protocolos de segurança adequados, tais como o login seguro por token JWT e a criptografia de informações confidenciais.  Além disso, possibilitará a modificação de preferências e informações de registro, fomentando um gerenciamento constante da identidade e garantindo uma grande customização da experiência do usuário.

\subsubsection{Módulo Educacional}

O objetivo deste módulo é incentivar a sensibilização e o aprendizado constante da população acerca da dengue por meio de uma plataforma multimídia com vídeos didáticos, textos explicativos e questionários interativos.  O material será estruturado em seções temáticas, tais como "Detecção de focos do Aedes aegypti", "Ciclo da dengue" e "Prevenção comunitária".  O avanço dos utilizadores poderá ser monitorado e premiado com vitórias, incorporando-se ao sistema de gamificação.

\subsubsection{Módulo de Gamificação}

Como já mencionado, a gamificação será o alicerce principal do engajamento do sistema.  O módulo incluirá classificações pessoais, classificações gerais e por bairro, resultados por atividades realizadas, tarefas individuais (como assistir a vídeos ou responder a questionários) e tarefas coletivas (como aprimorar a pontuação de uma área).  A participação ativa será premiada com pontos e medalhas.  
\par Também serão adicionadas recompensas inesperadas e desafios de duração limitada para incentivar o uso constante. Entre essas recompensas, podemos citar as recompenas virtuais no app como pontos de experiência, títulos, medalhas, entre outros. Pode-se incluir, também, recompensas reais com apoio de parceiros, como descontos em farmácias ou mercados locais, participação em sorteios e um certificado de participação social emitido digitalmente para usuários ativos, que podem ser útil para atividades extracurriculares ou inclusão em currículos.

\subsubsection{Módulo de Denúncias e Monitoramento de Áreas de Risco}

Este módulo permitirá que os usuários identifiquem focos de propagação do mosquito por meio de imagens, localização e descrição textual. Esses três elementos são essenciais para prevenir falsos alarmes e o uso desnecessário de recursos.   As queixas serão direcionadas aos órgãos competentes para validação e resposta.   Um mapa interativo mostrará as zonas de risco com base em dados geográficos, condições meteorológicas, queixas e registros de incidentes.   O módulo também irá notificar sobre o aumento do risco na região do usuário e fornecerá atualizações sobre a classificação do bairro, reforçando a natureza comunitária do aplicativo.

\subsubsection{Módulo de Monitoramento de Sintomas}

Este módulo, voltado para o bem-estar do usuário, coletará diariamente dados sobre sintomas que se assemelham à dengue.  Por meio de uma simples pesquisa, os usuários poderão comunicar febre, dor e outros sintomas.  Se houver fortes indícios de infecção, o sistema disparará alertas de suspeita, direcionando a procura por assistência médica e notificando as autoridades locais.  Este módulo também fornecerá informações para a avaliação epidemiológica do sistema.

\subsubsection{Módulo de Administração para Autoridades}

As autoridades de saúde pública terão acesso a uma interface administrativa contendo informações consolidadas sobre focos de dengue, estatísticas de sintomas e envolvimento da comunidade.  Elas terão a capacidade de confirmar denúncias, modificar o estado de ocorrências e importar informações de sistemas oficiais de monitoramento epidemiológico.  A meta é assegurar uma reação coordenada e fundamentada aos surtos, simplificando a decisão e a mobilização de recursos.


\subsection{Arquitetura de Aplicações}

O Zapp será construído com base em uma estrutura multicamadas e tambem Cliente-Servidor (como previamente mencionado), respeitando os princípios de separação de responsabilidades e escalabilidade horizontal.  Esta arquitetura tem como objetivo assegurar robustez, além de simplificar a manutenção e possíveis expansões futuras.  O sistema incluirá três camadas fundamentais: interface, lógica empresarial e armazenamento de dados, além de integrações com serviços externos.

\subsubsection{Camada de Apresentação (Frontend)}

A implementação do frontend será feita com \textit{frameworks} multiplataforma atuais, por meio \textit{React Native} em conjunto com \textit{TypeScript}, assegurando a compatibilidade com dispositivos Android e iOS.  Esta camada será encarregada de exibir a interface do usuário, recebendo interações e fornecendo feedback em tempo real.
Os componentes visuais abrangerão dashboards informativos, vídeos didáticos, formulários de denúncia e questionários de sintomas, bem como animações e notificações focadas na gamificação.  A interface será responsiva, de fácil acesso e concebida com o objetivo de facilitar a utilização e o envolvimento do público-alvo, mantendo os princípios de qualidade esperados e listados no documento Projeto de Interface.


\subsubsection{Camada de Lógica de Negócio (Backend)}

A lógica empresarial será concentrada numa API RESTful, construída com \textit{frameworks} sólidos usando \textit{Node.js} com Express.js. Esta camada funcionará como o elo entre o frontend e o banco de dados, tratando as solicitações, implementando as regras de negócio e assegurando a integridade das operações.
Esta camada cuidará de todas as normas relacionadas à pontuação, autenticação, gerenciamento de denúncias, geração de alertas e gerenciamento de permissões.  Também terá a tarefa de coordenar solicitações a serviços externos, tais como APIs meteorológicas e geográficas, e realizar verificações de segurança.

\subsubsection{Camada de Persistência (Banco de Dados)}

O armazenamento dos dados será feita em um banco de dados relacional usando o  \textit{PostgreSQL}, estruturado para armazenar todas as informações dos usuários, sintomas relatados, denúncias com georreferenciamento, rankings, conteúdos educacionais e estatísticas de uso.
Será adotado um modelo de dados normalizado, com suporte a consultas eficientes por localização e data, e com logs transacionais com a finalidade de auditoria e análise. A segurança dos dados será garantida através de criptografia em repouso e backups periódicos automatizados.

\subsubsection{Integrações com Serviços Externos}

A aplicação será implementada com APIs públicas e privadas para fornecer funcionalidades adicionais. Entre as integrações previstas no projeto inicial estão:

\begin{itemize}
    \item \textbf{APIs Climáticas:} usadas para obter dados de temperatura, umidade e chuvas, elementos estes que são chaves para a proliferação do mosquito e com isso, o aumento de casos da doença.
    \item \textbf{Serviços de Geolocalização:} como Google Maps ou OpenStreetMap, utilizados para exibir mapas para simplificar a apresentação de informações do aplicativo, além de auxiliar em ações como quando se faz uma denúncia.
    \item \textbf{Notificações Push:} envio de alertas em tempo real para usuários sobre riscos na região bem como atualizações sobre a situação atual da região, atualização dos rankings ou recompensas para manter o engajamento.
    \item \textbf{Bases Oficiais de Saúde:} importação de informações epidemiológicas e integração com entidades locais e regionais.
\end{itemize}

\subsubsection{Camada de Segurança e Autenticação}

A arquitetura da aplicação contará com mecanismos de segurança aplicados em todas as camadas. A autenticação será feita com tokens JWT e os dados sensíveis serão protegidos com criptografia AES. As comunicações entre o cliente e o servidor serão realizadas através de HTTPS com TLS 1.3, garantindo maior segurança e anonimidade. Também serão postas em prática políticas de controle de acesso com base no perfil do usuário, assegurando recursos específicos para os três diferentes níveis de usuários: administração, autoridade e cidadão.

\subsubsection{Escalabilidade e Infraestrutura}

O sistema foi desenvolvido com base no modelo MVC (Model-View-Controller), promovendo uma separação clara entre todas as camadas da aplicação, o que facilita a manutenção e escalabilidade da aplicação. O versionamento é realizado com o uso do Git com auxílio do GitHub, permitindo rastrear alterações, trabalhar em equipe de forma eficiente e manter um histórico organizado do desenvolvimento por meio da política de branches e code review. Para garantir a qualidade do código, são utilizados testes automatizados com a biblioteca Jest, que possibilita a execução contínua de testes unitários e de integração. 
\par Além disso, a aplicação opera sobre uma infraestrutura baseada em microsserviços e conteinerização, utilizando tecnologias como Docker e orquestração com Kubernetes. Isso possibilita a personalização de serviços conforme a demanda, reduzindo o tempo de implementação e simplificando o monitoramento. O sistema será hospedado em clouds como AWS, GCP ou Azure, com redundância geográfica e distribuição de carga equilibrada.

\newpage
\section{Conclusão}
Ao unir jogos educativos, alertas personalizados e mapas interativos, \textit{Zapp} convida a população a ser protagonista na luta contra a dengue, enquanto oferece às autoridades informações precisas para decisões estratégicas.

A escolha de uma arquitetura Cliente-Servidor e tecnologias escaláveis reflete um cuidado em manter o sistema vivo, sempre pronto para evoluir com novas funcionalidades, integrar-se a novas políticas públicas e de fácil manutenção. Recompensas virtuais, certificados de participação social e a integração com serviços locais não apenas motivam o engajamento, mas também reforçam laços comunitários.

No fim, o escopo do projeto está além de prevenir uma doença: propõe um modelo de como a inovação digital pode ser simples, envolvente e, acima de tudo, coletiva.

\newpage
\renewcommand{\refname}{Bibliografia}
\begin{thebibliography}{09}
\bibitem{ibge-2022} INSTITUTO BRASILEIRO DE GEOGRAFIA E ESTATÍSTICA (IBGE). 
\textbf{Censo Demográfico 2022: São José do Rio Preto}. 
2022. 
Disponível em: \url{https://www.ibge.gov.br/cidades-e-estados/sp/sao-jose-do-rio-preto.html}. 
Acesso em: 20 abr. 2025.

\bibitem{} %{title} %author name here
\textbf{}. %text here

\bibitem{} %{title} %author name here
\textbf{}. %text here

\end{thebibliography}
\end{document}
