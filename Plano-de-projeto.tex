\documentclass[a5paper, 12pt]{article}

\usepackage[portuges]{babel}
\usepackage[utf8]{inputenc}
\usepackage[T1]{fontenc}
\usepackage{times} % Fonte Times New Roman
\usepackage{setspace}
\usepackage{mathptmx} 
\onehalfspacing
\usepackage{amsmath}
\usepackage{indentfirst}
\usepackage{graphicx}
\usepackage{multicol,lipsum}
\usepackage{enumitem}
\usepackage{url}
\usepackage{booktabs}
\usepackage{array}
\usepackage{tocloft}
\usepackage{fancyhdr}
\usepackage{tabularx}   
\usepackage{longtable}
\usepackage{float}
\usepackage{booktabs}   
\usepackage{makecell}
\usepackage[a4paper, 
            left=3cm,    
            right=2cm,   
            top=3cm,    
            bottom=2cm   
            ]{geometry}
\usepackage{pgfgantt}
\usepackage{xcolor}
\usepackage{lscape}


\begin{document}
%\maketitle

\begin{titlepage}
	\begin{center}
	
	\begin{figure}[!ht]
	\centering
	\includegraphics[width=10cm]{dist/LogoTransparentePreto.png} \\ 
    \end{figure}

        
		\vspace{115pt}
        \textbf{\Huge{Plano de Projeto}}\\
		%\title{{\large{Título}}}
        
		\vspace{115pt}
        Carlos Eduardo Nogueira Silva \\
        Felipe Gomes da Silva \\
        Felipe Matheus Possari \\
        Matheus Thomé da Silva\\ 
        Santiago Pinheiro Martins \\
	\end{center}
	
	
	\vspace{1cm}
	\begin{center}
		\vspace{\fill}
		 Abril \\
		 2025
			\end{center}
\end{titlepage}
%%%%%%%%%%%%%%%%%%%%%%%%%%%%%%%%%%%%%%%%%%%%%%%%%%%%%%%%%%%


\newpage
\thispagestyle{empty}
\tableofcontents

\newpage
\pagestyle{fancy}

\fancyhead[L]{\thepage}
\fancyhead[C]{\nouppercase{\leftmark}}
\rhead{\includegraphics[width=2cm]{dist/LogoTransparentePreto.png}}
\fancyfoot[R]{}
\fancyfoot[C]{© 2025 - PuSystem}
\fancyfoot[L]{}
\setlength\headheight{26pt}

% % % % % % % % % % % % % % % % % % % % % % % % % % %

\newpage

\section{Introdução}
A relevância do planejamento em projetos de software é enfatizada por Pressman, que define a gerência de projetos como o conjunto de atividades de planejamento e acompanhamento das pessoas envolvidas, do produto e do processo ao longo de todo o ciclo de vida do software \cite{pressman2010}. Segundo o mesmo autor, um gerenciamento eficaz garante que o sistema atenda às expectativas do cliente e seja entregue dentro dos prazos e do orçamento previstos.  

Além disso, o Project Management Institute (PMI) aponta que, em média, “US\$ 122 milhões são desperdiçados para cada US\$ 1 bilhão investido devido ao baixo desempenho dos projetos” \cite{pmi-2016}, reforçando a importância de um planejamento rigoroso, da análise de riscos e do controle de cronograma desde o início do desenvolvimento.  

Entende-se, então, que os passos elaborados no documento a seguir são de extrema importância para o andamento do projeto, permitindo ao grupo de desenvolvimento, aos stakeholders e até mesmo à população uma melhor entrega, organização, otimização e entendimento do projeto como um todo.

\newpage
\section{Desenvolvimento}
\subsection{Organização do Projeto}
% descreve o modo como a equipe de desenvolvimento é organizada, as pessoas envolvidas e seus papéis na equipe
A equipe de desenvolvimento do projeto foi selecionada visando criar um grupo coeso, permitindo que os seus indivíduos trabalhem com eficiência em prol de entregar o melhor produto possível dentro do prazo e com bom custo-benefício. A seleção foi feita buscando criar um equilíbrio entre as habilidades técnicas (\textit{hard skills}) e personalidades complementares, facilitando assim o trabalho em equipe.

Levando em consideração o porte médio e a urgência do projeto, foi priorizada uma equipe pequena formada por desenvolvedores experientes e multifuncionais. Isto permite uma organização democrática e informal, com comunicação, troca de informação e tomada de decisão altamente eficaz. Desta forma, o líder de projeto está envolvido diretamente no desenvolvimento do software, e as tarefas são alocadas conforme as habilidades e experiências de cada membro. As decisões referentes ao projeto são tomadas através do consenso entre a equipe, com o líder técnico podendo usar sua experiência para guiá-las. A troca de informação, discussões e documentação do projeto é realizada através de uma \textit{wiki} particular e e-mails, permitindo que membros que trabalham remotamente se comuniquem efetivamente.

A metodologia de desenvolvimento usada é equilibrada, fazendo uso de muitos dos princípios da metodologia ágil e adaptando-se às necessidades do cliente. \textbf{Optou-se pelo framework Scrum} devido à sua ênfase em entregas incrementais, transparência e adaptabilidade. O Scrum favorece a auto-organização da equipe, permitindo que os membros definam suas próprias tarefas com base nas prioridades do \textit{Product Backlog}, enquanto mantêm ciclos de trabalho curtos (\textit{sprints} de 2 semanas). Isso alinha-se ao contexto do projeto, que exige respostas rápidas a mudanças regulatórias (como integrações com sistemas de saúde públicos) e à necessidade de validar funcionalidades com \textit{stakeholders} frequentemente [veja mais na seção \ref{sec:trabalho-cronograma}].

\newpage
\begin{longtable}[c]{@{} >{\raggedright\arraybackslash}p{4.5cm}>{\raggedright\arraybackslash}p{11cm} @{}}
  \caption{Equipe de desenvolvimento e papéis principais} \\
  \toprule
  \textbf{Membro e Cargo} & \textbf{Responsabilidades‑chave} \\
  \midrule
\endfirsthead

\multicolumn{2}{@{}l}{\textbf{Continuação da página anterior}} \\
  \toprule
  \textbf{Membro e Cargo} & \textbf{Responsabilidades‑chave} \\
  \midrule
\endhead

\bottomrule
\endfoot

\textbf{Carlos Eduardo} \\ \textit{Technical Lead \& Scrum Master} &
  Lidera a equipe técnica, sendo responsável pela definição da arquitetura do sistema, assegurando sua escalabilidade, segurança e performance.\
  Facilita a adoção e a prática de metodologias ágeis, coordenando as cerimônias ágeis e garantindo o alinhamento do time com os objetivos estratégicos do projeto.\
  Atua como ponto de comunicação entre a equipe técnica e as demais partes envolvidas, visando a eficiência no cumprimento de prazos e qualidade nas entregas. \\

\textbf{Felipe Gomes da Silva} \\ \textit{Backend \& Infrastructure Engineer} &
  Projetista e implementador da lógica de negócios no backend, desenvolvendo APIs altamente escaláveis e seguras, com foco em alta disponibilidade e desempenho.\
  Responsável pela infraestrutura do projeto, incluindo automação de deploys, integração contínua e gestão da nuvem e containers.\
  Aplica práticas avançadas de segurança e otimização de performance em toda a stack de backend. \\

\textbf{Felipe Possari} \\ \textit{Frontend Engineer \& PO Liaison} &
  Desenvolve interfaces de usuário altamente interativas e intuitivas, utilizando as mais avançadas tecnologias de frontend e garantindo a melhor experiência para o usuário (UX/UI).\
  Trabalha em estreita colaboração com as equipes de produto e stakeholders para traduzir requisitos de negócio em soluções técnicas viáveis e inovadoras.\
  \textit{É responsável pela comunicação contínua com stakeholders externos, assegurando que suas necessidades e expectativas sejam plenamente atendidas e alinhadas com os objetivos do projeto.} \\

\textbf{Santiago} \\ \textit{Data Scientist \& Analytics Engineer} &
  Aplica técnicas avançadas de ciência de dados, desenvolvendo modelos preditivos e algoritmos de machine learning para extrair insights valiosos a partir de grandes volumes de dados.\
  Lidera a criação de dashboards e relatórios analíticos para apoiar decisões estratégicas, além de garantir a integridade e qualidade dos dados ao longo do ciclo de vida do projeto.\
  Trabalha com grandes datasets, otimizando processos de limpeza, transformação e validação dos dados. \\

\textbf{Matheus Thomé} \\ \textit{QA Engineer \& DevOps Specialist} &
  Conduz o processo de garantia de qualidade do software, desenvolvendo uma abordagem robusta para testes automatizados, além de realizar auditorias manuais nas funcionalidades e no código.\
  Trabalha na integração contínua e automação de deploys, implementando pipelines eficientes e práticas de DevOps para garantir a agilidade e a qualidade nas entregas.\
  Além disso, é responsável por criar soluções para monitoramento e performance do sistema em produção, garantindo a estabilidade e escalabilidade. \\

\end{longtable}

A sinergia entre habilidades técnicas e papéis definidos no Scrum potencializa a eficiência da equipe. A combinação de expertise em segurança (Felipe Gomes), usabilidade (Felipe Possari), análise de dados (Santiago) e garantia de qualidade (Matheus Thomé), supervisionada pela liderança técnica de Carlos cria um ambiente propício para inovação e resposta rápida a desafios.


\subsection{Análise de Riscos}
% descreve possíveis riscos de projeto, a probabilidade de surgir tais riscos e as estratégias propostas para a sua redução

A avaliação de riscos consiste em um procedimento sistemático para detectar, avaliar e atenuar potenciais ameaças que possam afetar metas, projetos ou operações. Inclui a identificação de cenários desfavoráveis, a probabilidade de sua ocorrência e a severidade das consequências. Este método é fundamental em campos como engenharia, administração de projetos e segurança da informação, possibilitando decisões fundamentadas em provas. A abordagem mescla métodos qualitativos e quantitativos para destacar medidas preventivas e corretivas, minimizando incertezas e vulnerabilidades.

\subsubsection{Metodologia de Análise de Riscos}

Para o desenvolvimento adequado do projeto e propor estratégias com o fim de mitigar os principais riscos que podem comprometer o sucesso do projeto, adotamos a metodologia da análise qualitativa de riscos, na qual valores são atribuídos aos riscos para identificar o impacto potencial desses itens nos resultados possíveis. Assim, a avaliação é baseada nos seguintes critérios:

\begin{itemize}
    \item \textbf{Probabilidade (P):} Baixa (1), Média (2), Alta (3)
    \item \textbf{Impacto (I):} Baixo (1), Médio (2), Alto (3)
    \item \textbf{Nível de Risco (R):} R = P × I
    \item \textbf{Classificação:}
    \begin{itemize}
        \item Baixo (1 a 2)
        \item Moderado (3 a 4)
        \item Alto (6 a 8)
        \item Crítico (9)
    \end{itemize}
\end{itemize}

\subsubsection{Riscos Identificados}
A Tabela \ref{tab:risks} apresenta os principais riscos identificados no projeto, contemplando suas causas, potenciais impactos e estratégias iniciais de mitigação. Esta análise preliminar visa antecipar desafios críticos que poderiam comprometer prazos, custos ou qualidade do produto. 

\begin{table}
\caption{Riscos Identificados}
\label{tab:risks}
\begin{tabular}{|c|p{4cm}|c|c|c|p{5.7cm}|}
\hline
\textbf{ID} & \textbf{Risco} & \textbf{P} & \textbf{I} & \textbf{R} & \textbf{Estratégia} \\
\hline
R1 & Baixa adesão da população ao aplicativo & 2 & 3 & 6 (Alto) & Campanhas de conscientização, foco em UX e gamificação. \\
\hline
R2 & Denúncias falsas & 2 & 3 & 6 (Alto) & Validação por autoridades; penalidades para abuso. \\
\hline
R3 & Perda de dados ou vazamento & 2 & 3 & 6 (Alto) & Implementar criptografia AES-256, seguindo a LGPD. \\
\hline
R4 & Resistência de autoridades à adoção do sistema & 2 & 3 & 6 (Alto) & Envolver representantes da saúde pública nas etapas iniciais. Treinamentos e demonstração de benefícios. \\
\hline
R5 & Desinteresse após uso inicial (retenção de usuários) & 3 & 3 & 9 (Crítico) & Missões dinâmicas, notificações personalizadas, recompensas variadas e surpresas. \\
\hline
R6 & Difícil obtenção de dados climáticos e epidemiológicos em tempo real & 2 & 2 & 4 (Moderado) & Utilizar, bibliotecas, APIs e fontes públicas confiáveis, com fallback offline. \\
\hline
R7 & Desistência de membros da equipe durante o projeto & 1 & 3 & 3 (Moderado) & Garantir divisão de tarefas e documentação. \\
\hline
R8 & Falha no sistema de gamificação & 2 & 3 & 6 (Alto) & Testes rigorosos e feedback contínuo dos usuários\\
\hline
R9 & Dependência excessiva de autoridades para validar denúncias & 3 & 2 & 6 (Alto) & Automatizar validações com IA (ex.: análise de fotos) e capacitar equipes locais. \\
\hline
R10 & Falha na integração com sistemas oficiais de saúde & 2 & 3 & 6 (Alto) & APIs robustas, parcerias prévias com órgãos públicos, plano de contingência manual. \\
\hline
R11 & Conteúdo educacional ineficaz ou desatualizado & 1 & 2 & 2 (Baixo) & Revisão periódica por especialistas em saúde; feedback dos usuários. \\
\hline
R12 & Sobrecarga de servidores devido ao alto tráfego & 2 & 3 & 6 (Alto) & Infraestrutura escalável (cloud computing) e monitoramento em tempo real. \\
\hline
R13 & Viés em rankings (ex.: bairros mais pobres com menos engajamento) & 2 & 3 & 6 (Alto) & Adaptar critérios de pontuação por contexto socioeconômico e missões equilibradas. \\
\hline
R14 & Falha no monitoramento de sintomas  & 2 & 3 & 6 (Alto) & Revisão por médicos e alertas sobre limitações. \\
\hline
\end{tabular}
\smallskip
Elaborado pelos autores (2025).
\end{table}

\newpage
\subsubsection{Plano de Ação para Riscos Críticos}

Para os riscos classificados como \textbf{alto impacto} (R $\geq$ 6) e \textbf{críticos} (R = 9), serão adotadas as seguintes ações prioritárias:

\begin{itemize}
    \item \textbf{R1 (Baixa adesão):} Desenvolver MVP com foco em gamificação e realizar testes de usabilidade com moradores de diferentes perfis socioeconômicos.
    \item \textbf{R2 (Denúncias falsas):} Implementar sistema de verificação em duas etapas: análise automatizada de fotos por meio de IA e validação manual amostral.
    \item \textbf{R3 (Vazamento de dados):} Adotar autenticação OAuth2, criptografia AES-256 para dados sensíveis e auditorias de segurança bimestrais.
    \item \textbf{R4 (Resistência de autoridades):} Criar comitê consultivo com representantes da vigilância sanitária municipal desde a fase de prototipagem.
    \item \textbf{R5 (Desinteresse pós-uso):} Designar um "gamification owner" na equipe para atualizar semanalmente missões e recompensas surpresa.
    \item \textbf{R8 (Falhas na gamificação):} Implementar sistema A/B testing para mecânicas de pontuação e criar ambiente sandbox para testes antes do deploy.
    \item \textbf{R9 (Dependência de autoridades):} Capacitar agentes comunitários de saúde como validadores secundários via aplicativo dedicado.
    \item \textbf{R10 (Integração com sistemas):} Estabelecer parceria formal com a Secretaria Municipal de Saúde para acesso a APIs oficiais de dados epidemiológicos.
    \item \textbf{R12 (Sobrecarga de servidores):} Configurar auto-scaling na AWS com monitoramento contínuo via CloudWatch e plano de contingência para picos.
    \item \textbf{R13 (Viés em rankings):} Desenvolver algoritmo de ponderação que considere indicadores socioeconômicos do IBGE por região.
    \item \textbf{R14 (Falsos sintomas):} Integrar sistema com protocolos clínicos oficiais do Ministério da Saúde e incluir disclaimer médico nos alertas.
\end{itemize}

\subsubsection{Monitoramento Contínuo}
    Será realizada uma revisão trimestral do risk register pelo comitê que reúne a equipe técnica e os representantes da saúde pública de São José do Rio Preto com a finalidade de levantar os dados sobre os riscos e manter o aplicativo funcional e com boa performance. 
    Um relatório mensal de eficácia das ações mitigatórias também será feito, além da criação de dashboard de riscos para uma melhor compreensão do progresso da aplicação, contendo indicadores como: 
    \begin{itemize}
        \item Número de denúncias validadas/invalidadas
        \item Taxa de engajamento por bairro
        \item Tempo médio de resposta das autoridades
        \item Incidentes de segurança registrados
    \end{itemize}
    Além disso, haverá atualização dos níveis de risco após cada release importante do aplicativo.



\subsection{Requisitos de Apoio de Hardware e Software}
% descreve o hw e o sw de apoio exigidos para realizar o desenvolvimento. 
% Se o hw precisar ser comprado, incluir prazos de entrega e estimativas de preço

Para obter-se o melhor resultado na produção deste software, precisa-se analisar quais são os sistemas de apoio, tanto físicos quanto lógicos necessários. No geral, o desenvolvimento, pela parte da PuSystem, não precisara de uma grande massa de recursos, estes estarão ou armazenados em outros servidores, ou na própria borda do sistema, no usuário final. 

Estes, a seguir, são comportamentos esperados do sistema, baseados nos requisitos funcionais: 

\subsubsection{Desempenho e Qualidade}
\begin{itemize}[]
  \item Velocidade: Páginas principais devem carregar em até 200ms.
  \item Disponibilidade: Sistema online 24/7.
  \item Segurança: Criptografia avançada dos dados sensíveis.
  \item Usabilidade: Usuário deve se adaptar em até 2 horas através de uma UI simples e intuitiva.
\end{itemize}

O sucesso do projeto depende de uma infraestrutura adequada e da adoção de padrões técnicos claros. Para garantir eficiência, segurança e escalabilidade, foram definidos os seguintes requisitos:

\subsubsection{Tecnologia e Ferramentas}
\begin{itemize}[]
\item Front-end: React Native com TypeScript (para desenvolvimento multiplataforma).
\item Back-end: Node.js + Express.js, com PostgreSQL para armazenamento de dados.
\item Arquitetura: MVC (Model-View-Controller) para separação de responsabilidades.
\item Controle de Versão: Git + GitHub (com políticas de branches e code reviews).
\item Ferramentas Auxiliares: Docker para containerização e Jest para testes automatizados.
\end{itemize}

\subsubsection{Requisitos de Hardware}

\textbf{Para o Usuário (Celular)}
\begin{itemize}[]
\item Sistema Operacional: Android 10+, já que é compatível com 95\% dos dispositivos ativos, ou iOS 14+.
\item Hardware Mínimo: 3GB de RAM, 250MB de armazenamento livre, e suporte a GPS.
\item Conexão: 4G/LTE ou Wi-Fi estável.
\end{itemize}

\textbf{Para o Servidor (Escalado para 500 mil usuários)}
O calculo deste hardware eh baseado na ultima pesquisa do IBGE \cite{ibge-2022}
\begin{itemize}[]
\item Hardware:
- Processador: 32 núcleos (AMD EPYC ou Intel Xeon).
- Memória RAM: 512GB DDR4 ECC para alto throughput de requisições.
- Armazenamento: 20TB em SSD NVMe com redundância RAID 10.
\item Software:
- Sistema Operacional: Ubuntu Server 22.04 LTS para suporte de longo prazo.
- Banco de Dados: PostgreSQL 15 com replicação em cluster.
- Cache: Redis para otimização de consultas frequentes.
\item Rede:
- Banda larga: 2Gbps dedicada com balanceador de carga como NGINX.
- Firewall: Regras de segurança baseadas em IPS/IDS.
\end{itemize}

\subsubsection{Integrações externas}
\begin{itemize}[]
\item APIs de Terceiros:
- Clima: OpenWeatherMap para previsão em tempo real.
- Saúde: Integration with Apple HealthKit e Google Fit.
\item Conformidade Legal:
- LGPD: Auditorias trimestrais, criptografia AES-256 e registro de consentimento.
- Certificação SSL/TLS: Let's Encrypt ou soluções corporativas como por exemplo o DigiCert).
\end{itemize}

\subsubsection*{Desempenho e Segurança}
\begin{itemize}[]
\item Latência: Carregamento de páginas em até 1.5s em redes 4G.
\item Disponibilidade: SLA (Service Level Agreement) de 99.9\% 
\item Segurança: Autenticação em duas etapas (2FA) e varreduras de vulnerabilidades semanais.
\item Usabilidade: Taxa de retenção de usuários de até 70\% após 7 dias (métrica para validar UI/UX).
\end{itemize}

\subsubsection{Orçamento e aquisições}
\begin{table}[h]
\centering
\caption{Detalhamento de Custos para Implementação do Projeto}
\label{tab:custos}
\begin{tabular}{@{} >{\raggedright}p{2.5cm} r r p{3.5cm} @{}}
\toprule
\textbf{Item} & \textbf{Investimento (R\$)} & \textbf{Custo Anual (R\$)} & \textbf{Justificativa} \\
\midrule
Servidores (Hardware) & 100.000,00 & -- & Equipamentos de alta performance: 32 núcleos, 512GB RAM, 20TB SSD NVMe. \\
\addlinespace
Manutenção de Servidores & -- & 25.000,00 & Atualizações de segurança, reposição de peças e suporte técnico 24/7. \\
\addlinespace
Licenças de Software & -- & 15.000,00 & Licenças enterprise (PostgreSQL) e ferramentas de monitoramento (Grafana, Prometheus). \\
\addlinespace
Cloudflare/CDN & -- & 8.000,00 & Otimização de tráfego global e proteção contra ataques DDoS. \\
\bottomrule
\textbf{Total} & \textbf{100.000,00} & \textbf{48.000,00} & \\
\bottomrule
\end{tabular}

\smallskip
Elaborado pelos autores (2025).
\end{table}

Acredita-se, portanto, que os requisitos supracitados serão necessários para atender às demandas do produto, assim como as expectativas do contratante e, também, que os custos mencionados na tabela \ref{tab:custos} estão de acordo com o esperado.

\newpage


%%%%%%%%%%%%%%%%%%%%%%%%%%%%%%%%%%%%%%%%%%%%%%%%%%%%%%%%%%%%%%%%%%%%%%%%%%%%%%%%%%%%%%%%%%%%%%%%%%%%%%%%%%%%%%%%%%%%%%%%%%%%%%%%%%%%%%%%%%%%%%%%%%%%%%%%%%%%%%%%%%%%%%%%%%%%%%%

\subsection{Divisão de Trabalho e Cronograma do Projeto}
\label{sec:trabalho-cronograma}

Esta seção apresenta a organização do desenvolvimento do sistema com base nos requisitos funcionais definidos, detalhando as principais atividades, os responsáveis e os produtos entregáveis (marcos do projeto). A divisão foi feita por módulos correspondentes a cada requisito principal.

\subsubsection*{Requisito 1: Registro e Gestão de Usuários}
\textbf{Atividades:}
\begin{itemize}
  \item Definição da arquitetura de autenticação;
  \item Desenvolvimento da API de usuários;
  \item Criação da interface de cadastro e login;
  \item Testes de segurança e automação.
\end{itemize}
\textbf{Responsáveis:} Carlos Eduardo, Felipe Gomes, Felipe Possari, Matheus Thomé. \\
\textbf{Marco associado:} \textbf{M1 – Sistema de Usuários} (Sprint 3): Autenticação, cadastro e perfis operacionais.

\subsubsection*{Requisito 2: Módulo Educacional}
\textbf{Atividades:}
\begin{itemize}
  \item Modelagem do fluxo educacional;
  \item Implementação de APIs para conteúdo e quizzes;
  \item Desenvolvimento de interface interativa;
  \item Dashboards para acompanhamento do progresso.
\end{itemize}
\textbf{Responsáveis:} Carlos Eduardo, Felipe Gomes, Felipe Possari, Santiago, Matheus Thomé. \\
\textbf{Marco associado:} \textbf{M2 – Módulo Educacional} (Sprint 5): Conteúdo interativo e quizzes integrados.

\subsubsection*{Requisito 3: Sistema de Gamificação}
\textbf{Atividades:}
\begin{itemize}
  \item Definição das regras de gamificação;
  \item Backend para pontuação e ranking;
  \item Interface com elementos de engajamento;
  \item Métricas e testes de usabilidade.
\end{itemize}
\textbf{Responsáveis:} Carlos Eduardo, Felipe Gomes, Felipe Possari, Santiago, Matheus Thomé. \\
\textbf{Marco associado:} \textbf{M3 – Gamificação} (Sprint 10): Sistema de pontos, conquistas e feedback visual.

\subsubsection*{Requisito 4: Sistema de Denúncias e Áreas de Risco}
\textbf{Atividades:}
\begin{itemize}
  \item Integração com APIs de mapas;
  \item Interface para submissão e visualização de denúncias;
  \item Geração de heatmaps;
  \item Implementação de anonimização e segurança.
\end{itemize}
\textbf{Responsáveis:} Felipe Gomes, Felipe Possari, Santiago, Carlos Eduardo, Matheus Thomé. \\
\textbf{Marco associado:} \textbf{M4 – Denúncias e Áreas de Risco} (Sprint 13): Mapa interativo com denúncias georreferenciadas.

\subsubsection*{Requisito 5: Monitoramento de Sintomas}
\textbf{Atividades:}
\begin{itemize}
  \item Modelagem de dados clínicos;
  \item Backend para coleta e geração de alertas;
  \item Formulário dinâmico e validações no frontend;
  \item Testes com dados simulados.
\end{itemize}
\textbf{Responsáveis:} Santiago, Felipe Gomes, Felipe Possari, Carlos Eduardo, Matheus Thomé. \\
\textbf{Marco associado:} \textbf{M5 – Monitoramento de Sintomas} (Sprint 14): Histórico clínico e alertas implementados.

\subsubsection*{Requisito 6: Gerenciamento e Suporte a Autoridades}
\textbf{Atividades:}
\begin{itemize}
  \item Desenvolvimento de painel administrativo com níveis de acesso;
  \item Exportação e visualização de dados agregados;
  \item Geração de relatórios e dashboards analíticos.
\end{itemize}
\textbf{Responsáveis:} Carlos Eduardo, Felipe Gomes, Felipe Possari, Santiago, Matheus Thomé. \\
\textbf{Marco associado:} \textbf{M6 – Painel de Autoridades} (Sprint 17): Dashboard com relatórios e permissões diferenciadas.

\subsubsection*{Requisito 7: Infraestrutura, Testes e Lançamento}
\textbf{Atividades:}
\begin{itemize}
  \item Implementação de pipelines CI/CD;
  \item Monitoramento de desempenho e segurança;
  \item Auditoria final e documentação;
  \item Validação com stakeholders e preparação para produção.
\end{itemize}
\textbf{Responsáveis:} Felipe Gomes, Matheus Thomé, Carlos Eduardo, Felipe Possari. \\
\textbf{Marco associado:} \textbf{M7 – Lançamento Final} (Sprint 19): Sistema completo em produção e infraestrutura monitorada.

\subsubsection*{Cronograma Estimado}

A Tabela~\ref{tab:cronograma} apresenta a estimativa de tempo de desenvolvimento por requisito, baseada em ciclos de entrega contínua organizados em sprints quinzenais.

\begin{table}[H]
\centering
\caption{Duração Estimada por Requisito Principal}
\label{tab:cronograma}
\begin{tabular}{|l|c|c|}
\hline
\textbf{Requisito} & \textbf{Sprints} & \textbf{Duração (semanas)} \\
\hline
Registro e Gestão de Usuários & 2 a 3 & 4 a 6 \\
Módulo Educacional & 3  & 6 \\
Sistema de Gamificação & 4 a 5  & 8 a 10 \\
Sistema de Denúncias e Áreas de Risco & 3  & 6  \\
Monitoramento de Sintomas & 1  & 2  \\
Painel para Autoridades & 2 a 3  & 4 a 6  \\
Infraestrutura, Testes e Lançamento & 2  & 4 \\
\hline
\textbf{Total} & \textbf{17 a 20} & \textbf{34 a 40} \\
\hline
\end{tabular}
\smallskip

\raggedright
\textit{Fonte: Elaborado pelos autores (2025).}
\end{table}

\noindent \textbf{Considerações Complementares:}
\begin{itemize}
    \item \textbf{Duração Total do Projeto:} estimada entre \textbf{8,5 e 10 meses}.
    \item \textbf{Requisitos com maior complexidade:} o \textbf{Sistema de Gamificação} demanda mais esforço técnico e criativo, com duração prevista de até 10 semanas.
    \item \textbf{Requisitos com dependências externas:}
    \begin{itemize}
        \item O \textbf{Módulo Educacional} pode depender da validação de conteúdo especializado.
        \item O \textbf{Sistema de Denúncias} exige integração com serviços de mapas.
    \end{itemize}
    \item \textbf{Requisitos críticos:}
    \begin{itemize}
        \item O \textbf{Monitoramento de Sintomas} pode exigir ajustes caso integre dados reais de saúde.
        \item O \textbf{Lançamento Final} deve contemplar testes de carga, validações de segurança e testes de aceitação.
    \end{itemize}
\end{itemize}

\subsection{Marcos do Projeto}
\label{sec:marcos}

A Tabela~\ref{tab:marcos} relaciona os principais marcos do projeto aos respectivos sprints e produtos entregáveis, funcionando como referência para o acompanhamento do progresso.

\begin{table}[h]
\centering
\caption{Marcos do Projeto}
\label{tab:marcos}
\begin{tabular}{|l|c|l|}
\hline
\textbf{Marco} & \textbf{Sprint} & \textbf{Descrição do Produto Entregável} \\
\hline
M1 – Sistema de Usuários & Sprint 3 & Autenticação, cadastro e perfis operacionais \\
M2 – Módulo Educacional & Sprint 5 & Conteúdo educacional interativo e quizzes \\
M3 – Gamificação & Sprint 10 & Sistema de pontos, ranking e conquistas \\
M4 – Sistema de Denuncias & Sprint 13 & Submissão e visualização de denúncias no mapa \\
M5 – Monitoramento de Sintomas & Sprint 14 & Coleta e histórico de sintomas com alertas \\
M6 – Painel de Autoridades & Sprint 17 & Painel com dashboards e exportação de dados \\
M7 – Lançamento Final & Sprint 19 & Sistema completo em produção, com monitoramento \\
\hline
\end{tabular}
\end{table}

\noindent Para uma visualização temporal detalhada das fases, recursos e dependências, consulte o Diagrama \ref{fig:gantt-estilizado} a seguir.

\definecolor{barblue}{RGB}{153,204,254}
\definecolor{groupblue}{RGB}{51,102,254}
\definecolor{linkred}{RGB}{165,0,33}

\newpage
\begin{landscape}


\definecolor{barblue}{RGB}{153,204,254}
\definecolor{groupblue}{RGB}{51,102,254}
\definecolor{linkred}{RGB}{165,0,33}

\begin{figure}[htbp]
  \centering
  \begin{ganttchart}[
    canvas/.append style={fill=none, draw=black!5, line width=.100pt},
    hgrid style/.style={draw=black!5, line width=.100pt},
    vgrid={*1{draw=black!5, line width=.100pt}},
    title/.style={draw=none, fill=none},
    title label font=\bfseries\footnotesize,
    title label node/.append style={below=7pt},
    include title in canvas=false,
    bar label font=\mdseries\small\color{black!70},
    bar label node/.append style={left=2cm},
    bar/.append style={draw=none, fill=barblue},
    bar incomplete/.append style={fill=barblue},
    group incomplete/.append style={fill=groupblue},
    group left shift=0,
    group right shift=0,
    group height=.5,
    group peaks tip position=0,
    group label node/.append style={left=.5cm},
    link/.style={-latex, line width=1.5pt, linkred}
  ]{1}{19}
  
  \gantttitle[
    title label node/.append style={below left=7pt and 3pt}
  ]{SPRINTS:\quad1}{1}
  \gantttitlelist{2,...,19}{1} \\[grid]
  
  % Requisito 1: Usuários
  \ganttgroup[progress=0]{Requisito 1: Registro e Gestão de Usuários}{1}{3} \\
  \ganttbar[progress=0,name=M1]{\textbf{M1} Sistema de Usuários}{3}{3} \\[grid]
  
  % Requisito 2: Educacional
  \ganttgroup[progress=25]{Requisito 2: Módulo Educacional}{4}{5} \\
  \ganttbar[progress=25,name=M2]{\textbf{M2} Módulo Educacional}{5}{5} \\[grid]
  
  % Requisito 3: Gamificação
  \ganttgroup[progress=50]{Requisito 3: Gamificação}{6}{10} \\
  \ganttbar[progress=50,name=M3]{\textbf{M3} Sistema de Gamificação}{10}{10} \\[grid]
  
  % Requisito 4: Denúncias
  \ganttgroup[progress=75]{Requisito 4: Denúncias e Áreas de Risco}{11}{13} \\
  \ganttbar[progress=75,name=M4]{\textbf{M4} Sistema de Denúncias}{13}{13} \\[grid]
  
  % Requisito 5: Sintomas
  \ganttgroup[progress=90]{Requisito 5: Monitoramento de Sintomas}{14}{14} \\
  \ganttbar[progress=90,name=M5]{\textbf{M5} Monitoramento de Sintomas}{14}{14} \\[grid]
  
  % Requisito 6: Autoridades
  \ganttgroup[progress=95]{Requisito 6: Painel de Autoridades}{15}{17} \\
  \ganttbar[progress=95,name=M6]{\textbf{M6} Painel de Autoridades}{17}{17} \\[grid]
  
  % Requisito 7: Infraestrutura
  \ganttgroup[progress=100]{Requisito 7: Infraestrutura e Lançamento}{18}{19} \\
  \ganttbar[progress=100,name=M7]{\textbf{M7} Lançamento Final}{19}{19}
  
  \end{ganttchart}
  \caption{Diagrama de Gantt com Marcos e Progresso}
  \label{fig:gantt-estilizado}
\end{figure}
\end{landscape}
\newpage


%%%%%%%%%%%%%%%%%%%%%%%%%%%%%%%%%%%%%%%%%%%%%%%%%%%%%%%%%%%%%%%%%%%%%%%%%%%%%%%%%%%%%%%%%%%%%%%%%%%%%%%%%%%%%%%%%%%%%%%%%%%%%%%%%%%%%%%%%%%%%%%%%%%%%%%%%%%%%%%%%%%%%%%%%%%%%%%%%%%%%%%%%%%%%%%%%%%%%%%%%%%%%%%%%%%%%%%%%%%%%%%%%%%%%%%


\subsection{Mecanismos de Monitoramento e Elaboração de Relatórios}

Para garantir o acompanhamento contínuo do projeto e a tomada de decisões fundamentadas, a PuSystem adotará mecanismos sistemáticos de monitoramento e elaboração de relatórios, organizados em três frentes principais: desenvolvimento e qualidade, uso e engajamento, e infraestrutura e conformidade.

\subsubsection{Desenvolvimento e Qualidade}

Durante o desenvolvimento, ferramentas de gestão ágil, como Jira e Trello, serão utilizadas para controlar as tarefas da equipe. Relatórios semanais ao final de cada sprint apresentarão o status das tarefas (concluídas, pendentes, bloqueadas) e o progresso geral do projeto. Além disso, a equipe coletará métricas como taxa de defeitos por sprint, tempo médio de resolução de bugs e cobertura de testes automatizados. Esses dados serão compilados em relatórios ao final de cada sprint, promovendo a melhoria contínua do processo de desenvolvimento.

\subsubsection{Uso e Engajamento}

Após o lançamento do sistema, plataformas de análise como Google Analytics e Firebase monitorarão o comportamento dos usuários. Indicadores como número de usuários ativos, taxa de retenção, volume de denúncias realizadas, participação em missões e consumo de conteúdos educacionais serão analisados e apresentados em relatórios trimestrais. Para as autoridades públicas, um painel administrativo fornecerá relatórios atualizados sobre focos de dengue denunciados, evolução das áreas de risco e distribuição de sintomas reportados, apoiando ações estratégicas de saúde pública.

\subsubsection{Infraestrutura e Conformidade}

O sistema contará com soluções de monitoramento em tempo real, como Prometheus e Grafana, para acompanhar a disponibilidade, latência e desempenho dos servidores. Relatórios semanais de incidentes de infraestrutura serão elaborados, permitindo a identificação de gargalos e a adoção de medidas preventivas. Além disso, auditorias trimestrais de segurança e conformidade com a Lei Geral de Proteção de Dados (LGPD) serão realizadas, com relatórios que incluirão resultados de varreduras de vulnerabilidades e avaliações das práticas de segurança implementadas.

Todos os relatórios serão organizados e disponibilizados em uma wiki interna segura, acessível aos membros da equipe e representantes das autoridades públicas, promovendo transparência, rápida disseminação de informações e sustentação de um processo contínuo de melhoria da PuSystem.

\newpage
\section{Conclusão}
O planejamento detalhado apresentado neste documento demonstra a viabilidade técnica e organizacional do projeto, em conformidade com as melhores práticas de administração de software propostas por Pressman e com os critérios de eficácia do PMI. A organização em sprints com objetivos claros (M1–M7), apoiada por um gráfico de Gantt dinâmico e mecanismos de supervisão constante, estabelece uma estrutura sólida para minimizar os riscos críticos identificados - desde obstáculos técnicos como a integração de sistemas até questões sociais como a participação da população.

A adoção do Scrum potencializa a capacidade de resposta da equipe a mudanças, enquanto a análise quantitativa de riscos (com 14 fatores críticos mapeados e priorizados) garantiu proatividade na mitigação de ameaças. A combinação de ferramentas modernas – como autenticação OAuth2, criptografia AES-256 e infraestrutura escalável em nuvem – com parcerias estratégicas (como o comitê consultivo com autoridades sanitárias) assegura não apenas a entrega técnica, mas a relevância social da solução.

Os indicadores de desempenho estabelecidos (taxa de retenção >70\%, SLA de 99.9\% e carregamento de páginas em <1.5s) funcionarão como termômetros objetivos do sucesso, enquanto os relatórios trimestrais de segurança e engajamento sustentarão a melhoria contínua. Este projeto consolida-se, assim, como um modelo replicável de como planejamento rigoroso, métodos ágeis e monitoramento baseado em dados podem transformar desafios complexos de saúde pública em soluções tecnológicas eficazes, reduzindo o desperdício de recursos e maximizando o impacto social.


\newpage
\renewcommand{\refname}{Bibliografia}
\begin{thebibliography}{09}
\bibitem{ibge-2022} INSTITUTO BRASILEIRO DE GEOGRAFIA E ESTATÍSTICA (IBGE). 
\textbf{Censo Demográfico 2022: São José do Rio Preto}. 
2022. 
Disponível em: \url{https://www.ibge.gov.br/cidades-e-estados/sp/sao-jose-do-rio-preto.html}. 
Acesso em: 20 abr. 2025.

\bibitem{pressman2010} PRESSMAN, R. S. \textit{Software Engineering: A Practitioner’s Approach}. 7. ed. McGraw‑Hill, 2010.

\bibitem{pmi-2016} PROJECT MANAGEMENT INSTITUTE (PMI). \textbf{Pulse of the Profession™ 2016: The High Cost of Low Performance}. 2016. Disponível em: \url{https://www.pmi.org/-/media/pmi/documents/public/pdf/learning/thought-leadership/pulse/pulse-of-the-profession-2016.pdf#:~:text=More%20critical%20is%20the%20money,percent%20increase%20over%20last%20year}. Acesso em: 02 abr. 2025.

\end{thebibliography}
\end{document}
